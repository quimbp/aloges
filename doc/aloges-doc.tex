\documentclass[12pt,a4paper]{report}

% ------------------------------------------------------------
% Packages
% ------------------------------------------------------------
\usepackage[utf8]{inputenc}      % Encoding
\usepackage[T1]{fontenc}         % Font encoding
\usepackage{lmodern}             % Latin Modern font
\usepackage{geometry}            % Page margins
\geometry{margin=1in}
\usepackage{setspace}            % Line spacing
\usepackage{graphicx}            % Figures
\usepackage{caption}             % Better captions
\usepackage{subcaption}          % Subfigures
\usepackage{amsmath, amssymb}    % Math
\usepackage{hyperref}            % Clickable refs
\usepackage{listings}            % Code listings
\usepackage{xcolor}              % Colors
\usepackage{fancyhdr}            % Headers and footers
\usepackage{biblatex}            % Bibliography
\addbibresource{references.bib}  % Bib file

% ------------------------------------------------------------
% Fortran style
% ------------------------------------------------------------
\lstdefinelanguage{Fortran90}%
  {morekeywords={module, use, implicit, none, integer, parameter, real, end},%
   sensitive=true,%
   morecomment=[l]!%
  }

\lstset{%
  language=Fortran90,
  basicstyle=\ttfamily\small,
  keywordstyle=\color{blue}\bfseries,
  commentstyle=\color{gray}\itshape,
  numbers=none,
  showstringspaces=false,
  frame=single,
  rulecolor=\color{black!30}
}

% ------------------------------------------------------------
% Header & Footer
% ------------------------------------------------------------
\pagestyle{fancy}
\fancyhf{}
\fancyhead[L]{Technical Report Title}
\fancyhead[R]{\leftmark}
\fancyfoot[C]{\thepage}

% ------------------------------------------------------------
% Title Page
% ------------------------------------------------------------
\title{
    \vspace{2in}
    \Huge \textbf{The Aloges Project} \\
    \vspace{0.5in}
    \Large 1. Library contents\\
    \vspace{1in}
    \textbf{Joaquim Ballabrera} \\
    Department of Ocean Physics and Technology / ICM - CSIC \\
    \vfill
    %\includegraphics[width=0.3\textwidth]{logo.png} \\
    \vfill
    \today
}
\author{}
\date{}

% ------------------------------------------------------------
% Document
% ------------------------------------------------------------
\begin{document}

\maketitle
\pagenumbering{roman}
\tableofcontents
\listoffigures
\listoftables

\newpage
\pagenumbering{arabic}

% ------------------------------------------------------------
% Sections
% ------------------------------------------------------------
\chapter{Introduction}
Provide background, motivation, and scope of the report.

\chapter{Literature Review}
Summarize existing work related to your report.

\chapter{Modules}
Describe methods, models, algorithms, or experiments used.

\section{Module module\_types}

\paragraph{} 
The \texttt{module\_types} module defines a small set of standard Fortran 
numerical type parameters and constants to ensure consistency across the Aloges utilities. To improve protability, it imports kind definitions from the intrinsic \texttt{iso\_fortran\_env} module and assigns 
them to more convenient aliases: \texttt{i4} for 32-bit integers, \texttt{sp} for 
single-precision real numbers, and \texttt{dp} for double-precision real numbers. 
In addition, it provides the parameter \texttt{maxlen}, which specifies a maximum 
character string length. All the modules descrived here use this module to
facilitate portability, readability and the uniform use of data precision throughout the whole library.

\paragraph{}
To include this module, the library modules invoke this module with the statement
\begin{lstlisting}
use module_types
\end{lstlisting}
\noindent at the beginning of the code. Then, to define a double-precision, real variable in a portable, architecture-independent manner, the user can simply use:

\begin{lstlisting}
real(dp) a        ! Double precision variable.
\end{lstlisting}

\section{Module module\_constants}

\paragraph{} 
The \texttt{module\_constants} module provides a comprehensive set of constants and utility values commonly used in geophysical and mathematical computations. 
It includes logical constants, mathematical constants (integers, fractions, $\pi$, $e$, conversions between degrees and radians), complex numbers, as well as representations for NaN and Infinity. 
This organization ensures portability, readability, and consistency across numerical code, especially when working with double-precision and single-precision real numbers in Fortran. 
The module also contains a structured type \texttt{type\_constants} that encapsulates physical and geophysical constants such as standard gravity, Earth properties, solar irradiance, and universal physical constants.

\paragraph{} 
To aid clarity and usability, the constants are categorized into logical, mathematical, and physical/geophysical constants. 
Mathematical constants are defined as \texttt{parameter} values, ensuring they are compile-time constants, whereas NaN and Infinity values are initialized at runtime using dedicated subroutines (\texttt{set\_nan}, \texttt{set\_inf}). 
Physical constants are stored within the \texttt{type\_constants} derived type, which is instantiated as \texttt{constants} for immediate use. 
This module allows for convenient reference to fundamental values and simplifies calculations in scientific simulations.

% ------------------------------------------------------------
% Table: Mathematical constants
% ------------------------------------------------------------
\begin{table}[h!]
\centering
\caption{Mathematical constants defined in \texttt{module\_constants}}
\begin{tabular}{llc}
\hline
\textbf{Name} & \textbf{Value} & \textbf{Description} \\
\hline
\texttt{pi}      & 3.141592653589793$_{dp}$ & $\pi$ \\
\texttt{two\_pi} & 2$\pi$ & Two times $\pi$ \\
\texttt{half\_pi}& 0.5$\pi$ & Half of $\pi$ \\
\texttt{inv\_pi} & $1/\pi$ & Reciprocal of $\pi$ \\
\texttt{euler}& 2.718281828459045$_{dp}$ & Base of natural logarithm \\
\texttt{deg2rad} & $\pi/180$ & Degrees to radians conversion \\
\texttt{rad2deg} & $180/\pi$ & Radians to degrees conversion \\
\texttt{half}    & 0.5 & Fraction one-half \\
\texttt{quarter} & 0.25 & Fraction one-quarter \\
\texttt{minus}   & -1.0 & Negative one \\
\texttt{zero}    & 0.0 & Zero \\
\texttt{one}     & 1.0 & One \\
\texttt{two}     & 2.0 & Two \\
\hline
\end{tabular}
\end{table}

% ------------------------------------------------------------
% Table: Physical constants
% ------------------------------------------------------------
\begin{table}[h!]
\centering
\caption{Physical and geophysical constants defined in \texttt{type\_constants}}
\begin{tabular}{llc}
\hline
\textbf{Name} & \textbf{Value} & \textbf{Description} \\
\hline
\texttt{Earth\_Gravity}     & 9.80665$_{dp}$ & Standard gravity [m/s$^2$] \\
\texttt{Earth\_Omega}       & 7.2921159$\times10^{-5}$$_{dp}$ & Earth's angular velocity [rad/s] \\
\texttt{Earth\_Radius}      & 6.371315$\times10^6$$_{dp}$ & Mean Earth radius [m] \\
\texttt{Earth\_Mass}         & 5.9722$\times10^{24}$$_{dp}$ & Earth mass [kg] \\
\texttt{Earth\_Declination} & 23.446$_{dp}$ & Orbital declination [deg] \\
\texttt{Solar\_Constant}    & 1365.2$_{dp}$ & Solar irradiance at 1 AU [W/m$^2$] \\
\texttt{G}                   & 6.67430$\times10^{-11}$$_{dp}$ & Gravitational constant [m$^3$/kg/s$^2$] \\
\texttt{sigma\_SB}           & 5.670374419$\times10^{-8}$$_{dp}$ & Stefan-Boltzmann constant [W/m$^2$/K$^4$] \\
\texttt{k\_B}                & 1.380649$\times10^{-23}$$_{dp}$ & Boltzmann constant [J/K] \\
\texttt{R\_gas}              & 8.314462618$_{dp}$ & Universal gas constant [J/mol/K] \\
\texttt{Avogadro\_Number}                & 6.02214076$\times10^{23}$$_{dp}$ & Avogadro number [1/mol] \\
\hline
\end{tabular}
\end{table}

% ------------------------------------------------------------
% Example usage
% ------------------------------------------------------------
\paragraph{Example usage: Mathematical constants}
\begin{verbatim}
  ! Compute circumference of a circle
  use module_constants
  real(dp) :: radius, circumference
  radius = 6371.0_dp
  circumference = two_pi * radius
\end{verbatim}

\paragraph{Example usage: Physical constants}
\begin{verbatim}
  ! Compute gravitational force between Earth and a satellite
  use module_constants
  real(dp) :: F, m_sat, r
  m_sat = 1000.0_dp
  r = constants%Earth_Radius + 400000.0_dp  ! 400 km altitude
  F = constants%G * constants%Earth_Mass * m_sat / r**2
\end{verbatim}



\section{Module module\_time}

\paragraph{}
The \texttt{module\_time} module provides utilities for handling dates, times, calendars, and astronomical calculations. It includes a derived type \texttt{type\_date} and supports various calendars (gregorian, noleap). The module also includes functions for converting between dates and numerical representations, calculating sunrise/sunset, and estimating solar insolation.

\subsection{\texttt{check\_calendar(calendar)}}
\begin{itemize}
\item \textbf{Purpose}: Validates and normalizes the calendar string.
\item \textbf{Arguments}:
\begin{itemize}
\item \texttt{calendar} (\texttt{character(len=*)}, \texttt{intent(inout)}): Calendar name to check.
\end{itemize}
\item \textbf{Example}:
\begin{lstlisting}[language=Fortran]
call check_calendar(cal)  ! Converts 'standard' to 'gregorian'
\end{lstlisting}
\end{itemize}

\subsection{\texttt{check\_units(units)}}
\begin{itemize}
\item \textbf{Purpose}: Validates and normalizes time units.
\item \textbf{Arguments}:
\begin{itemize}
\item \texttt{units} (\texttt{character(len=*)}, \texttt{intent(inout)}): Time units to check.
\end{itemize}
\item \textbf{Example}:
\begin{lstlisting}[language=Fortran]
call check_units('sec')  ! Converts to 'seconds'
\end{lstlisting}
\end{itemize}

\subsection{\texttt{days\_before\_month(year, month, calendar)}}
\begin{itemize}
\item \textbf{Purpose}: Returns the number of days in the year before the start of the given month.
\item \textbf{Arguments}:
\begin{itemize}
\item \texttt{year} (\texttt{integer}, \texttt{intent(in)}): Year.
\item \texttt{month} (\texttt{integer}, \texttt{intent(in)}): Month.
\item \texttt{calendar} (\texttt{character(len=*)}, \texttt{intent(in)}, \texttt{optional}): Calendar type.
\end{itemize}
\item \textbf{Returns}: \texttt{integer} number of days.
\item \textbf{Example}:
\begin{lstlisting}[language=Fortran]
ndays = days_before_month(2023, 3, 'gregorian')
\end{lstlisting}
\end{itemize}

\subsection{\texttt{days\_before\_year(year, calendar)}}
\begin{itemize}
\item \textbf{Purpose}: Returns the number of days before January 1st of the given year.
\item \textbf{Arguments}:
\begin{itemize}
\item \texttt{year} (\texttt{integer}, \texttt{intent(in)}): Year.
\item \texttt{calendar} (\texttt{character(len=*)}, \texttt{intent(in)}, \texttt{optional}): Calendar type.
\end{itemize}
\item \textbf{Returns}: \texttt{integer} number of days.
\item \textbf{Example}:
\begin{lstlisting}[language=Fortran]
ndays = days_before_year(2023, 'gregorian')
\end{lstlisting}
\end{itemize}

\subsection{\texttt{days\_in\_month(year, month, calendar)}}
\begin{itemize}
\item \textbf{Purpose}: Returns the number of days in the given month and year.
\item \textbf{Arguments}:
\begin{itemize}
\item \texttt{year} (\texttt{integer}, \texttt{intent(in)}): Year.
\item \texttt{month} (\texttt{integer}, \texttt{intent(in)}): Month.
\item \texttt{calendar} (\texttt{character(len=*)}, \texttt{intent(in)}, \texttt{optional}): Calendar type.
\end{itemize}
\item \textbf{Returns}: \texttt{integer} number of days.
\item \textbf{Example}:
\begin{lstlisting}[language=Fortran]
ndays = days_in_month(2023, 2, 'gregorian')  ! Returns 28 or 29
\end{lstlisting}
\end{itemize}

\subsection{\texttt{isleap(year, calendar)}}
\begin{itemize}
\item \textbf{Purpose}: Checks if the given year is a leap year.
\item \textbf{Arguments}:
\begin{itemize}
\item \texttt{year} (\texttt{integer}, \texttt{intent(in)}): Year.
\item \texttt{calendar} (\texttt{character(len=*)}, \texttt{intent(in)}, \texttt{optional}): Calendar type.
\end{itemize}
\item \textbf{Returns}: \texttt{logical} true if leap year.
\item \textbf{Example}:
\begin{lstlisting}[language=Fortran]
if (isleap(2024)) then
  ! Do something
end if
\end{lstlisting}
\end{itemize}

\subsection{\texttt{split\_units(units, time\_units, RefDate)}}
\begin{itemize}
\item \textbf{Purpose}: Splits a time units string into time units and reference date.
\item \textbf{Arguments}:
\begin{itemize}
\item \texttt{units} (\texttt{character(len=*)}, \texttt{intent(in)}): Input string (e.g., "days since 2000-01-01").
\item \texttt{time\_units} (\texttt{character(len=20)}, \texttt{intent(out)}): Extracted time units.
\item \texttt{RefDate} (\texttt{character(len=20)}, \texttt{intent(out)}): Extracted reference date.
\end{itemize}
\item \textbf{Example}:
\begin{lstlisting}[language=Fortran]
call split_units('days since 2000-01-01', tu, rd)
\end{lstlisting}
\end{itemize}

\subsection{\texttt{unit\_conversion\_factor(units)}}
\begin{itemize}
\item \textbf{Purpose}: Returns the conversion factor to seconds for the given time units.
\item \textbf{Arguments}:
\begin{itemize}
\item \texttt{units} (\texttt{character(len=*)}, \texttt{intent(in)}): Time units.
\end{itemize}
\item \textbf{Returns}: \texttt{real(dp)} conversion factor.
\item \textbf{Example}:
\begin{lstlisting}[language=Fortran]
factor = unit_conversion_factor('days')  ! Returns 86400.0
\end{lstlisting}
\end{itemize}

\subsection{\texttt{ymd2ord(year, month, day, calendar)}}
\begin{itemize}
\item \textbf{Purpose}: Converts year, month, day to an ordinal day number.
\item \textbf{Arguments}:
\begin{itemize}
\item \texttt{year} (\texttt{integer}, \texttt{intent(in)}): Year.
\item \texttt{month} (\texttt{integer}, \texttt{intent(in)}): Month.
\item \texttt{day} (\texttt{integer}, \texttt{intent(in)}): Day.
\item \texttt{calendar} (\texttt{character(len=*)}, \texttt{intent(in)}, \texttt{optional}): Calendar type.
\end{itemize}
\item \textbf{Returns}: \texttt{integer} ordinal day.
\item \textbf{Example}:
\begin{lstlisting}[language=Fortran]
ord = ymd2ord(2023, 1, 1)  ! Returns 1
\end{lstlisting}
\end{itemize}

\subsection{\texttt{ord2ymd(nn, year, month, day, calendar)}}
\begin{itemize}
\item \textbf{Purpose}: Converts an ordinal day number to year, month, day.
\item \textbf{Arguments}:
\begin{itemize}
\item \texttt{nn} (\texttt{integer}, \texttt{intent(in)}): Ordinal day.
\item \texttt{year} (\texttt{integer}, \texttt{intent(out)}): Year.
\item \texttt{month} (\texttt{integer}, \texttt{intent(out)}): Month.
\item \texttt{day} (\texttt{integer}, \texttt{intent(out)}): Day.
\item \texttt{calendar} (\texttt{character(len=*)}, \texttt{intent(in)}, \texttt{optional}): Calendar type.
\end{itemize}
\item \textbf{Example}:
\begin{lstlisting}[language=Fortran]
call ord2ymd(1, y, m, d)  ! y=1, m=1, d=1
\end{lstlisting}
\end{itemize}

\subsection{\texttt{date\_is(y, m, d, hh, mm, ss, calendar)}}
\begin{itemize}
\item \textbf{Purpose}: Creates a \texttt{type\_date} instance.
\item \textbf{Arguments}:
\begin{itemize}
\item \texttt{y, m, d} (\texttt{integer}, \texttt{intent(in)}): Year, month, day.
\item \texttt{hh, mm, ss} (\texttt{integer}, \texttt{intent(in)}, \texttt{optional}): Hour, minute, second.
\item \texttt{calendar} (\texttt{character(len=*)}, \texttt{intent(in)}, \texttt{optional}): Calendar type.
\end{itemize}
\item \textbf{Returns}: \texttt{type\_date} instance.
\item \textbf{Example}:
\begin{lstlisting}[language=Fortran]
d = date_is(2023, 1, 1, 12, 0, 0)
\end{lstlisting}
\end{itemize}

\subsection{\texttt{date\_set(p, y, m, d, hh, mm, ss, cal)}}
\begin{itemize}
\item \textbf{Purpose}: Sets the values of a \texttt{type\_date} instance.
\item \textbf{Arguments}:
\begin{itemize}
\item \texttt{p} (\texttt{type\_date}, \texttt{intent(inout)}): Date instance.
\item \texttt{y, m, d} (\texttt{integer}, \texttt{intent(in)}): Year, month, day.
\item \texttt{hh, mm, ss} (\texttt{integer}, \texttt{intent(in)}, \texttt{optional}): Hour, minute, second.
\item \texttt{cal} (\texttt{character(len=*)}, \texttt{intent(in)}, \texttt{optional}): Calendar type.
\end{itemize}
\item \textbf{Example}:
\begin{lstlisting}[language=Fortran]
call date_set(d, 2023, 1, 1, 12, 0, 0)
\end{lstlisting}
\end{itemize}

\subsection{\texttt{date\_iso(date, Z)}}
\begin{itemize}
\item \textbf{Purpose}: Returns an ISO 8601 string representation of the date.
\item \textbf{Arguments}:
\begin{itemize}
\item \texttt{date} (\texttt{type\_date}, \texttt{intent(in)}): Date instance.
\item \texttt{Z} (\texttt{character(len=*)}, \texttt{intent(in)}, \texttt{optional}): Timezone specifier.
\end{itemize}
\item \textbf{Returns}: \texttt{character(len=25)} ISO string.
\item \textbf{Example}:
\begin{lstlisting}[language=Fortran]
str = d%iso('Z')  ! Returns "2023-01-01T12:00:00Z"
\end{lstlisting}
\end{itemize}

\subsection{\texttt{julday(year, month, day)}}
\begin{itemize}
\item \textbf{Purpose}: Computes the Julian day number.
\item \textbf{Arguments}:
\begin{itemize}
\item \texttt{year, month, day} (\texttt{integer}, \texttt{intent(in)}): Date components.
\end{itemize}
\item \textbf{Returns}: \texttt{integer} Julian day.
\item \textbf{Example}:
\begin{lstlisting}[language=Fortran]
jd = julday(2023, 1, 1)
\end{lstlisting}
\end{itemize}

\subsection{\texttt{caldat(julian, year, month, day)}}
\begin{itemize}
\item \textbf{Purpose}: Converts a Julian day number to calendar date.
\item \textbf{Arguments}:
\begin{itemize}
\item \texttt{julian} (\texttt{integer}, \texttt{intent(in)}): Julian day.
\item \texttt{year, month, day} (\texttt{integer}, \texttt{intent(out)}): Date components.
\end{itemize}
\item \textbf{Example}:
\begin{lstlisting}[language=Fortran]
call caldat(2459945, y, m, d)
\end{lstlisting}
\end{itemize}

\subsection{\texttt{date\_now(now)}}
\begin{itemize}
\item \textbf{Purpose}: Sets the current date and time.
\item \textbf{Arguments}:
\begin{itemize}
\item \texttt{now} (\texttt{type\_date}, \texttt{intent(inout)}): Date instance to set.
\end{itemize}
\item \textbf{Example}:
\begin{lstlisting}[language=Fortran]
call now%now()
\end{lstlisting}
\end{itemize}

\subsection{\texttt{date2jd(date)}}
\begin{itemize}
\item \textbf{Purpose}: Converts a \texttt{type\_date} to Julian day (with fractional part).
\item \textbf{Arguments}:
\begin{itemize}
\item \texttt{date} (\texttt{type\_date}, \texttt{intent(in)}): Date instance.
\end{itemize}
\item \textbf{Returns}: \texttt{real(dp)} Julian day.
\item \textbf{Example}:
\begin{lstlisting}[language=Fortran]
jd = d%jd()
\end{lstlisting}
\end{itemize}

\subsection{\texttt{jd2date(jd)}}
\begin{itemize}
\item \textbf{Purpose}: Converts a Julian day to a \texttt{type\_date}.
\item \textbf{Arguments}:
\begin{itemize}
\item \texttt{jd} (\texttt{real(dp)}, \texttt{intent(in)}): Julian day.
\end{itemize}
\item \textbf{Returns}: \texttt{type\_date} instance.
\item \textbf{Example}:
\begin{lstlisting}[language=Fortran]
d = jd2date(2459945.5)
\end{lstlisting}
\end{itemize}

\subsection{\texttt{date\_increment(date, days, hours, minutes, seconds)}}
\begin{itemize}
\item \textbf{Purpose}: Increments (or decrements) a date by the given amounts.
\item \textbf{Arguments}:
\begin{itemize}
\item \texttt{date} (\texttt{type\_date}, \texttt{intent(in)}): Base date.
\item \texttt{days, hours, minutes, seconds} (\texttt{integer}, \texttt{intent(in)}, \texttt{optional}): Offsets.
\end{itemize}
\item \textbf{Returns}: \texttt{type\_date} new date.
\item \textbf{Example}:
\begin{lstlisting}[language=Fortran]
new_d = d%timedelta(days=1, hours=12)
\end{lstlisting}
\end{itemize}

\subsection{\texttt{num2date(time, units, calendar)}}
\begin{itemize}
\item \textbf{Purpose}: Converts a numerical time to a \texttt{type\_date}.
\item \textbf{Arguments}:
\begin{itemize}
\item \texttt{time} (\texttt{real(dp)}, \texttt{intent(in)}): Numerical time.
\item \texttt{units} (\texttt{character(len=*)}, \texttt{intent(in)}, \texttt{optional}): Time units.
\item \texttt{calendar} (\texttt{character(len=*)}, \texttt{intent(in)}, \texttt{optional}): Calendar type.
\end{itemize}
\item \textbf{Returns}: \texttt{type\_date} instance.
\item \textbf{Example}:
\begin{lstlisting}[language=Fortran]
d = num2date(86400.0, 'seconds since 2000-01-01')
\end{lstlisting}
\end{itemize}

\subsection{\texttt{date2num(date, units)}}
\begin{itemize}
\item \textbf{Purpose}: Converts a \texttt{type\_date} to a numerical time.
\item \textbf{Arguments}:
\begin{itemize}
\item \texttt{date} (\texttt{type\_date}, \texttt{intent(in)}): Date instance.
\item \texttt{units} (\texttt{character(len=*)}, \texttt{intent(in)}, \texttt{optional}): Time units.
\end{itemize}
\item \textbf{Returns}: \texttt{real(dp)} numerical time.
\item \textbf{Example}:
\begin{lstlisting}[language=Fortran]
t = date2num(d, 'days since 2000-01-01')
\end{lstlisting}
\end{itemize}

\subsection{\texttt{time\_transform(time, units, calendar, units\_new, calendar\_new)}}
\begin{itemize}
\item \textbf{Purpose}: Transforms a time array from one units/calendar to another.
\item \textbf{Arguments}:
\begin{itemize}
\item \texttt{time} (\texttt{real(dp)}, \texttt{dimension(:)}, \texttt{intent(in)}): Input time array.
\item \texttt{units, calendar} (\texttt{character(len=*)}, \texttt{intent(in)}): Original units and calendar.
\item \texttt{units\_new, calendar\_new} (\texttt{character(len=*)}, \texttt{intent(in)}): Target units and calendar.
\end{itemize}
\item \textbf{Returns}: \texttt{real(dp)}, \texttt{dimension(size(time))} transformed time array.
\item \textbf{Example}:
\begin{lstlisting}[language=Fortran]
t_new = time_transform(t_old, 'days', 'gregorian', 'hours', 'noleap')
\end{lstlisting}
\end{itemize}

\subsection{\texttt{hour2sec(hh, mm, ss, sec)}}
\begin{itemize}
\item \textbf{Purpose}: Converts hours, minutes, seconds to total seconds.
\item \textbf{Arguments}:
\begin{itemize}
\item \texttt{hh, mm, ss} (\texttt{integer}, \texttt{intent(in)}): Time components.
\item \texttt{sec} (\texttt{integer}, \texttt{intent(out)}): Total seconds.
\end{itemize}
\item \textbf{Example}:
\begin{lstlisting}[language=Fortran]
call hour2sec(1, 30, 0, sec)  ! sec = 5400
\end{lstlisting}
\end{itemize}

\subsection{\texttt{sec2hour(sec, hh, mm, ss)}}
\begin{itemize}
\item \textbf{Purpose}: Converts total seconds to hours, minutes, seconds.
\item \textbf{Arguments}:
\begin{itemize}
\item \texttt{sec} (\texttt{integer}, \texttt{intent(in)}): Total seconds.
\item \texttt{hh, mm, ss} (\texttt{integer}, \texttt{intent(out)}): Time components.
\end{itemize}
\item \textbf{Example}:
\begin{lstlisting}[language=Fortran]
call sec2hour(5400, hh, mm, ss)  ! hh=1, mm=30, ss=0
\end{lstlisting}
\end{itemize}

\subsection{\texttt{dms2dec(gg, mm, ss, deg)}}
\begin{itemize}
\item \textbf{Purpose}: Converts degrees, minutes, seconds to decimal degrees.
\item \textbf{Arguments}:
\begin{itemize}
\item \texttt{gg, mm, ss} (\texttt{integer}, \texttt{intent(in)}): DMS components.
\item \texttt{deg} (\texttt{real(dp)}, \texttt{intent(out)}): Decimal degrees.
\end{itemize}
\item \textbf{Example}:
\begin{lstlisting}[language=Fortran]
call dms2dec(45, 30, 0, deg)  ! deg = 45.5
\end{lstlisting}
\end{itemize}

\subsection{\texttt{dec2dms(deg, gg, mm, ss)}}
\begin{itemize}
\item \textbf{Purpose}: Converts decimal degrees to degrees, minutes, seconds.
\item \textbf{Arguments}:
\begin{itemize}
\item \texttt{deg} (\texttt{real(dp)}, \texttt{intent(in)}): Decimal degrees.
\item \texttt{gg, mm, ss} (\texttt{integer}, \texttt{intent(out)}): DMS components.
\end{itemize}
\item \textbf{Example}:
\begin{lstlisting}[language=Fortran]
call dec2dms(45.5, gg, mm, ss)  ! gg=45, mm=30, ss=0
\end{lstlisting}
\end{itemize}

\subsection{\texttt{SunRise\_and\_SunSet(lon, lat, date, hSunrise, hSunset)}}
\begin{itemize}
\item \textbf{Purpose}: Computes sunrise and sunset times for a given location and date.
\item \textbf{Arguments}:
\begin{itemize}
\item \texttt{lon, lat} (\texttt{real(dp)}, \texttt{intent(in)}): Longitude and latitude (degrees).
\item \texttt{date} (\texttt{type\_date}, \texttt{intent(in)}): Date.
\item \texttt{hSunrise, hSunset} (\texttt{real(dp)}, \texttt{intent(out)}): Sunrise and sunset times (seconds from midnight).
\end{itemize}
\item \textbf{Example}:
\begin{lstlisting}[language=Fortran]
call SunRise_and_SunSet(2.0, 41.0, d, sunrise, sunset)
\end{lstlisting}
\end{itemize}

\subsection{\texttt{daily\_insolation(lat, day)}}
\begin{itemize}
\item \textbf{Purpose}: Estimates daily solar insolation for a given latitude and day of year.
\item \textbf{Arguments}:
\begin{itemize}
\item \texttt{lat} (\texttt{real(dp)}, \texttt{intent(in)}): Latitude (radians).
\item \texttt{day} (\texttt{real(dp)}, \texttt{intent(in)}): Day of year.
\end{itemize}
\item \textbf{Returns}: \texttt{real(dp)} insolation (W/m²).
\item \textbf{Example}:
%\begin{lstlisting}[language=Fortran]
%Qsw = daily_insolation(0.7, 180.0)  ! Latitude ~40°, day 180
%\end{lstlisting}
\end{itemize}

\section{Notes}
\begin{itemize}
\item The module uses double precision (\texttt{real(dp)}) for numerical calculations.
\item Calendar support includes 'gregorian' (leap years) and 'noleap' (365-day years).
\item Time units can be 'seconds', 'minutes', 'hours', or 'days'.
\item The module includes internal helper functions for astronomical calculations (e.g., \texttt{SunDeclination}, \texttt{EquationOfTime}).
\end{itemize}

This module is robust and suitable for a wide range of time and date handling tasks in scientific applications.


\chapter{Results and Discussion}
Present results, use figures and tables, and discuss their meaning.

\chapter{Conclusion and Future Work}
Summarize contributions, limitations, and possible future directions.

% ------------------------------------------------------------
% Appendix (optional)
% ------------------------------------------------------------
\appendix
\chapter{Additional Data}
Put supplementary material here.

% ------------------------------------------------------------
% References
% ------------------------------------------------------------
\printbibliography

\end{document}

