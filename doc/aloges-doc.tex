\documentclass[12pt,a4paper]{report}

% ------------------------------------------------------------
% Packages
% ------------------------------------------------------------
\usepackage[utf8]{inputenc}      % Encoding
\usepackage[T1]{fontenc}         % Font encoding
\usepackage{lmodern}             % Latin Modern font
\usepackage{geometry}            % Page margins
\geometry{margin=1in}
\usepackage{setspace}            % Line spacing
\usepackage{graphicx}            % Figures
\usepackage{caption}             % Better captions
\usepackage{subcaption}          % Subfigures
\usepackage{amsmath, amssymb}    % Math
\usepackage{hyperref}            % Clickable refs
\usepackage{listings}            % Code listings
\usepackage{xcolor}              % Colors
\usepackage{fancyhdr}            % Headers and footers
\usepackage{biblatex}            % Bibliography
\addbibresource{references.bib}  % Bib file

% ------------------------------------------------------------
% Fortran style
% ------------------------------------------------------------
\lstdefinelanguage{Fortran90}%
  {morekeywords={module, use, implicit, none, integer, parameter, real, end},%
   sensitive=true,%
   morecomment=[l]!%
  }

\lstset{%
  language=Fortran90,
  basicstyle=\ttfamily\small,
  keywordstyle=\color{blue}\bfseries,
  commentstyle=\color{gray}\itshape,
  numbers=none,
  showstringspaces=false,
  frame=single,
  rulecolor=\color{black!30}
}

% ------------------------------------------------------------
% Header & Footer
% ------------------------------------------------------------
\pagestyle{fancy}
\fancyhf{}
\fancyhead[L]{Technical Report Title}
\fancyhead[R]{\leftmark}
\fancyfoot[C]{\thepage}

% ------------------------------------------------------------
% Title Page
% ------------------------------------------------------------
\title{
    \vspace{2in}
    \Huge \textbf{The Aloges Project} \\
    \vspace{0.5in}
    \Large 1. Library contents\\
    \vspace{1in}
    \textbf{Joaquim Ballabrera} \\
    Department of Ocean Physics and Technology / ICM - CSIC \\
    \vfill
    %\includegraphics[width=0.3\textwidth]{logo.png} \\
    \vfill
    \today
}
\author{}
\date{}

% ------------------------------------------------------------
% Document
% ------------------------------------------------------------
\begin{document}

\maketitle
\pagenumbering{roman}
\tableofcontents
\listoffigures
\listoftables

\newpage
\pagenumbering{arabic}

% ------------------------------------------------------------
% Sections
% ------------------------------------------------------------
\chapter{Introduction}
Provide background, motivation, and scope of the report.

\chapter{Literature Review}
Summarize existing work related to your report.

\chapter{Modules}
Describe methods, models, algorithms, or experiments used.

\section{Module module\_types}

\paragraph{} 
The \texttt{module\_types} module defines a small set of standard Fortran 
numerical type parameters and constants to ensure consistency across the Aloges utilities. To improve protability, it imports kind definitions from the intrinsic \texttt{iso\_fortran\_env} module and assigns 
them to more convenient aliases: \texttt{i4} for 32-bit integers, \texttt{sp} for 
single-precision real numbers, and \texttt{dp} for double-precision real numbers. 
In addition, it provides the parameter \texttt{maxlen}, which specifies a maximum 
character string length. All the modules descrived here use this module to
facilitate portability, readability and the uniform use of data precision throughout the whole library.

\paragraph{}
To include this module, the library modules invoke this module with the statement
\begin{lstlisting}
use module_types
\end{lstlisting}
\noindent at the beginning of the code. Then, to define a double-precision, real variable in a portable, architecture-independent manner, the user can simply use:

\begin{lstlisting}
real(dp) a        ! Double precision variable.
\end{lstlisting}

\section{Module module\_constants}

\paragraph{} 
The \texttt{module\_constants} module provides a comprehensive set of constants and utility values commonly used in geophysical and mathematical computations. 
It includes logical constants, mathematical constants (integers, fractions, $\pi$, $e$, conversions between degrees and radians), complex numbers, as well as representations for NaN and Infinity. 
This organization ensures portability, readability, and consistency across numerical code, especially when working with double-precision and single-precision real numbers in Fortran. 
The module also contains a structured type \texttt{type\_constants} that encapsulates physical and geophysical constants such as standard gravity, Earth properties, solar irradiance, and universal physical constants.

\paragraph{} 
To aid clarity and usability, the constants are categorized into logical, mathematical, and physical/geophysical constants. 
Mathematical constants are defined as \texttt{parameter} values, ensuring they are compile-time constants, whereas NaN and Infinity values are initialized at runtime using dedicated subroutines (\texttt{set\_nan}, \texttt{set\_inf}). 
Physical constants are stored within the \texttt{type\_constants} derived type, which is instantiated as \texttt{constants} for immediate use. 
This module allows for convenient reference to fundamental values and simplifies calculations in scientific simulations.

% ------------------------------------------------------------
% Table: Mathematical constants
% ------------------------------------------------------------
\begin{table}[h!]
\centering
\caption{Mathematical constants defined in \texttt{module\_constants}}
\begin{tabular}{llc}
\hline
\textbf{Name} & \textbf{Value} & \textbf{Description} \\
\hline
\texttt{pi}      & 3.141592653589793$_{dp}$ & $\pi$ \\
\texttt{two\_pi} & 2$\pi$ & Two times $\pi$ \\
\texttt{half\_pi}& 0.5$\pi$ & Half of $\pi$ \\
\texttt{inv\_pi} & $1/\pi$ & Reciprocal of $\pi$ \\
\texttt{euler}& 2.718281828459045$_{dp}$ & Base of natural logarithm \\
\texttt{deg2rad} & $\pi/180$ & Degrees to radians conversion \\
\texttt{rad2deg} & $180/\pi$ & Radians to degrees conversion \\
\texttt{half}    & 0.5 & Fraction one-half \\
\texttt{quarter} & 0.25 & Fraction one-quarter \\
\texttt{minus}   & -1.0 & Negative one \\
\texttt{zero}    & 0.0 & Zero \\
\texttt{one}     & 1.0 & One \\
\texttt{two}     & 2.0 & Two \\
\hline
\end{tabular}
\end{table}

% ------------------------------------------------------------
% Table: Physical constants
% ------------------------------------------------------------
\begin{table}[h!]
\centering
\caption{Physical and geophysical constants defined in \texttt{type\_constants}}
\begin{tabular}{llc}
\hline
\textbf{Name} & \textbf{Value} & \textbf{Description} \\
\hline
\texttt{Earth\_Gravity}     & 9.80665$_{dp}$ & Standard gravity [m/s$^2$] \\
\texttt{Earth\_Omega}       & 7.2921159$\times10^{-5}$$_{dp}$ & Earth's angular velocity [rad/s] \\
\texttt{Earth\_Radius}      & 6.371315$\times10^6$$_{dp}$ & Mean Earth radius [m] \\
\texttt{Earth\_Mass}         & 5.9722$\times10^{24}$$_{dp}$ & Earth mass [kg] \\
\texttt{Earth\_Declination} & 23.446$_{dp}$ & Orbital declination [deg] \\
\texttt{Solar\_Constant}    & 1365.2$_{dp}$ & Solar irradiance at 1 AU [W/m$^2$] \\
\texttt{G}                   & 6.67430$\times10^{-11}$$_{dp}$ & Gravitational constant [m$^3$/kg/s$^2$] \\
\texttt{sigma\_SB}           & 5.670374419$\times10^{-8}$$_{dp}$ & Stefan-Boltzmann constant [W/m$^2$/K$^4$] \\
\texttt{k\_B}                & 1.380649$\times10^{-23}$$_{dp}$ & Boltzmann constant [J/K] \\
\texttt{R\_gas}              & 8.314462618$_{dp}$ & Universal gas constant [J/mol/K] \\
\texttt{Avogadro\_Number}                & 6.02214076$\times10^{23}$$_{dp}$ & Avogadro number [1/mol] \\
\hline
\end{tabular}
\end{table}

% ------------------------------------------------------------
% Example usage
% ------------------------------------------------------------
\paragraph{Example usage: Mathematical constants}
\begin{verbatim}
  ! Compute circumference of a circle
  use module_constants
  real(dp) :: radius, circumference
  radius = 6371.0_dp
  circumference = two_pi * radius
\end{verbatim}

\paragraph{Example usage: Physical constants}
\begin{verbatim}
  ! Compute gravitational force between Earth and a satellite
  use module_constants
  real(dp) :: F, m_sat, r
  m_sat = 1000.0_dp
  r = constants%Earth_Radius + 400000.0_dp  ! 400 km altitude
  F = constants%G * constants%Earth_Mass * m_sat / r**2
\end{verbatim}


\chapter{Results and Discussion}
Present results, use figures and tables, and discuss their meaning.

\chapter{Conclusion and Future Work}
Summarize contributions, limitations, and possible future directions.

% ------------------------------------------------------------
% Appendix (optional)
% ------------------------------------------------------------
\appendix
\chapter{Additional Data}
Put supplementary material here.

% ------------------------------------------------------------
% References
% ------------------------------------------------------------
\printbibliography

\end{document}

